% !TEX TS-program = lualatex
\documentclass[english]{article}

% Test program: Apply the 'selnolig' package, with
% 'english' language option set, to a list of English
% words which contain various character pairs that
% should not be ligated. The list of English words is 
% in the companion file 'selnolig-english-wordlist.tex'.
%
% Author: Mico Loretan (loretan dot mico at gmail dot com)
% Date: 2013/09/21

% Check first that we're running Lua(La)TeX.
\usepackage{ifluatex}
\ifluatex\else
  \typeout{===============================================}
  \typeout{The file selnolig-english-test.tex must be     }
  \typeout{compile using lualatex. Exiting immediately.   }
  \typeout{===============================================}
  \endinput
\fi

\usepackage[margin=1in]{geometry}



\usepackage{fancyvrb}

\usepackage{fontspec}

\defaultfontfeatures{%
     Ligatures={TeX,Common,Rare},
     Numbers  = OldStyle}

\setmainfont[ 
    FeatureFile    = gpp-ft.fea,
    ItalicFont     = {Garamond Premier Pro Italic},
    BoldItalicFont = {Garamond Premier Pro Italic},
    BoldFont       = {Garamond Premier Pro}]
    {Garamond Premier Pro}

\setmonofont[Scale=MatchLowercase,
             Ligatures=NoCommon]
            {Consolas}

\newfontfamily\ebg[ Scale = MatchLowercase,
    Ligatures      = {TeX,Common,Rare,Historic,Contextual},
    Contextuals = Alternate,
    ItalicFont     = {EB Garamond 12 Italic},
    ItalicFeatures = {Scale = MatchLowercase}]
    {EB Garamond}

\usepackage[document]{ragged2e}
\usepackage{babel}
\usepackage[broadf,hdlig,zwnj]{selnolig}
\debugon

\usepackage{sectsty}
  \allsectionsfont{\mdseries}

%\usepackage{showhyphens}

\setlength\parindent{0pt}
\parskip=0.3\baselineskip
\usepackage{multicol}
  \setlength\columnseprule{.4pt}
\title{selnolig: List of English test words\\ 
(Package version: \selnoligpackageversion \quad
\selnoligpackagedate)}
\author{\null}
\date{}

\righthyphenmin=2 % set this to either 3 (normal) or 2


\begin{document}
\maketitle

\begin{tabular}{@{}*{10}{l}}
Appearance of ``common'' ligatures 
   &ff &fi &fl &ffi &ffl &ft & \mbox{fj} & {\ebg\mbox{fk}} 
   & \emph{fr}\\
Sample words with these ligatures
   &off &fit &fly &office &baffle &often & fjord 
   &{\ebg Kafka} &\emph{from}\\
\end{tabular}

\bigskip

\begin{tabular}{@{}*{15}{l}}
Appearance of ``discretionary'' (``rare'') ligatures 
 & st & ct & sp  
 & \emph{as} & {\ebg\em es} & \emph{is, th, us}
 & \emph{at} & \emph{et} & {\ebg\emph{sk}}& \emph{ll} & \emph{\uselig{ij}, st, ta}\\
Sample words with these ligatures
 & stay & act & spy 
 & \emph{was} & {\ebg\em sees}& \emph{is\kern0ptthmus} & \emph{cat} & \emph{net} 
 & {\ebg\em ask} & \emph{ill} & \emph{rijsttafel}\\
\end{tabular}


\bigskip

\makeatletter
\begin{tabular}{@{}ll}
Package options and other parameters:\\
\ \ \ \ \ Extent of suppressing f-ligatures: basic or broadf?  & \if@broadfset broadf \else basic \fi \\
\ \ \ \ \ Value of \texttt{\textbackslash righthyphenmin} parameter & \the\righthyphenmin\\
\end{tabular}
\makeatother

\bigskip

\begin{multicols}{2}
\input selnolig-english-wordlist
\end{multicols}


\end{document}
